\documentclass{article}
% preamble (Packages importieren)

\author{Eduard Scherer}
\title{Development \& Construction of an Autonomous Path-Following Drone}
\date{\today}
\usepackage[utf8]{inputenc}

\usepackage[
backend=biber,
style=numeric,
sorting=ynt
]{biblatex}

\addbibresource{references.bib} %Imports bibliography file


% Dokument
\begin{document}
\maketitle
	\section{Introduction}
	do at the end of the writing process
	\section{Personal Motivation}
	rather short
	\section{Literature Review}
	\subsection{general software considerations}
	why I choose the software
	(work around the same topic that already exists)
	\section{Methodology}
	(only what has been done + which parts are needed and why I choose them)
	\subsection{parts}
	\begin{itemize}
		\item \textbf{Kakute H7}, comes with betaflight
	\end{itemize}
	\subsection{soldering}
	
	\subsection{ardupilot}
	\subsubsection{Ground Station}
	\textbf{ground station} = software running on ground-based computer, transmit data to the UAV and can control it
	\begin{itemize}
		\item \textbf{Mission Planner} widely used and has a wiki(Open Source)
		\begin{itemize}
			\item 
		\end{itemize}
		\item \textbf{Mavproxy}, for Linux used for code developers, written in Python(Open Source)
		\item some considerations for Smartphone (might be useful for connection between Raspi and ground)
	\end{itemize}
	downloading of latest stable Ardupilot firmware for Kakute H7 fc	
	
	loading Ardupilot firmware onto Fc with STM32 CubeProgrammer by connecting the Fc in DFU(Direct Firmware Update) Mode with the computer(quick explanation box for STM32 and DFU mode)
	
	reconnecting Fc with Comp. 
	opening Mission Planner and connecting to Fc seeing first Yaw measurements 
	\section{Results}
	\section{Discussion and Outlook}
	\section{Conclusion}
	
	\section{References}
	\printbibliography[
	heading=bibintoc,
	title={Bibliography}
	]	
	\section{Table of Figures}
	(short table on which every figure description with the page number is listed)
\end{document}
